% !TeX encoding = UTF-8
% !TeX spellcheck = de_DE
% !TeX root = ../thesis.tex

\chapter{Formeln} \label{sec:Formeln}

Formeln können mit den in \texttt{preamble/symbolliste} definierten Formelzeichen eingefügt werden. So werden die Formelzeichen direkt korrekt verlinkt, siehe \cref{eq:formel1,eq:formel2}. Beachte auch den Befehl \texttt{\textbackslash vind}, der für Vektoren einen leicht eingerückten Index bereitstellt. Weitere Befehle für die Darstellung von Formeln sind in \texttt{dissertation-font-package.sty} definiert, z.B.\ \texttt{\textbackslash tind} für Textindizes. Die Verwendung dieser Befehle ergibt ein typografisch deutlich besseres Bild, vergleiche \cref{eq:formel1,eq:formel3}. Es sollte sowieso ein wenig auf die Typografie geachtet werden, so ist z.B.\ wie folgt zu setzen \texttt{z.B.\textbackslash} -- mit \textit{Backslash}, sonst stimmen die Abstände nicht.

\begin{align}
	\label{eq:formel1}
	\gls{x}\vind{\gls{ist}} & = \gls{x}\vind{\gls{soll}} + \gls{err}    \\
	\label{eq:formel3}
	\gls{x}_{\gls{ist}}     & = \gls{x}_{\gls{soll}} + \gls{err}        \\
	\label{eq:formel2}
	\gls{phi}_{\gls{ist}}   & = \gls{phi}_{\gls{soll}} + \gls{deltaphi}
\end{align}

\blindtext[3]


\chapter{Weiteres}

\section{Tabellen}
\label{sec:HFS_Wahrnehmung}

Eine Tabelle könnte wie folgt aussehen, siehe \cref{tab:programme}.

\begin{table}[hbt]
	\centering
	\caption{Programme zur Erstellung von Literaturverzeichnissen.}
	\label{tab:programme}
	\begin{tabular}{lll}
		\toprule
		Programm & quelloffen und frei & Latex-geeignet \\ \midrule
		JabRef   & ja                  & ja             \\ \bottomrule
	\end{tabular}
\end{table}

\blindtext[3]


\section{Abbildungen}
\label{sec:abbildungen}

Abbildungen sollten unter Verwendung des hier bereitgestellten Befehls \texttt{\textbackslash includeFig} eingebunden werden. So wird je nach Wahl von \texttt{\textbackslash INCLUDETIKZ} entweder das PDF oder die TikZ-Datei eingebunden. Zwar dauert das Einbinden der TikZ-Dateien deutlich länger, jedoch ist auch die Dateigröße geringer und Formelzeichen und Symbole werden verlinkt. Eine Abbildung kann dann wie folgt aussehen, siehe \cref{img:Dissertationsprozess}. Die Skripte zur Erstellung der TikZ-Grafiken, z.B.\ \texttt{Script\textbackslash{}compile\textbackslash{}figures.sh}, erzeugen die Grafiken übrigens Parallel in bis zu 16 Threads.

\blindtext[5]

\begin{figure}[hbt]
	\centering
	\includeFig{Dissertationsprozess}{tikz/dissertationsprozess}
	\caption{Prozess zur Erstellung einer Dissertation}
	\label{img:Dissertationsprozess}
\end{figure}

\blindtext[20]
