% !TeX encoding = UTF-8
% !TeX spellcheck = de_DE
% !TeX root = ../dissertation.tex

\chapter{Zusammenfassung und Ausblick}
\label{sec:Zusammenfassung}

Insgesamt ist diese Latex-Vorlage ganz toll, aber es gibt auch noch Punkte, die verbessert werden können.

Dies hier ist ein Blindtext zum Testen von Textausgaben. Wer diesen Text liest, ist selbst schuld.
Der Text gibt lediglich den Grauwert der Schrift an. Ist das wirklich so? Ist es gleichgültig, obich schreibe:
„Dies ist ein Blindtext“ oder „Huardest gefburn“? Kjift – mitnichten! Ein Blindtextbietet mir wichtige Informationen. 
An ihm messe ich die Lesbarkeit einer Schrift, ihre Anmutung,wie harmonisch die Figuren zueinander stehen und prüfe, 
wie breit oder schmal sie läuft. EinBlindtext sollte möglichst viele verschiedene Buchstaben enthalten und in der 
Originalsprachegesetzt sein. Er muss keinen Sinn ergeben, sollte aber lesbar sein. Fremdsprachige Texte wie„Lorem ipsum“ 
dienen nicht dem eigentlichen Zweck, da sie eine falsche Anmutung vermitteln.Dies hier ist ein Blindtext zum Testen von 
Textausgaben. Wer diesen Text liest, ist selbst schuld.Der Text gibt lediglich den Grauwert der Schrift an. Ist das 
wirklich so? Ist es gleichgültig, obich schreibe: „Dies ist ein Blindtext“ oder „Huardest gefburn“? Kjift – mitnichten! 
Ein Blindtextbietet mir wichtige Informationen. An ihm messe ich die Lesbarkeit einer Schrift, ihre Anmutung,wie 
harmonisch die Figuren zueinander stehen und prüfe, wie breit oder schmal sie läuft. EinBlindtext sollte möglichst viele 
verschiedene Buchstaben enthalten und in der Originalsprachegesetzt sein. Er muss keinen Sinn ergeben, sollte aber lesbar 
sein. Er muss keinen Sinn ergeben, sollte aber lesbar sein. Fremdsprachige Texte wie„Lorem ipsum“ 
dienen nicht dem eigentlichen Zweck, da sie eine falsche Anmutung vermitteln.Dies hier ist ein Blindtext zum Testen von 
Textausgaben. Wer diesen Text liest, ist selbst schuld.Der Text gibt lediglich den Grauwert der Schrift an. Ist das 
wirklich so? Ist es gleichgültig, obich schreibe: „Dies ist ein Blindtext“ oder „Huardest gefburn“? Kjift – mitnichten! 
Ein Blindtextbietet mir wichtige Informationen. An ihm messe ich die Lesbarkeit einer Schrift, ihre Anmutung,wie 
harmonisch die Figuren zueinander stehen und prüfe, wie breit oder schmal sie läuft. EinBlindtext sollte möglichst viele 
verschiedene Buchstaben enthalten und in der Originalsprachegesetzt sein. Er muss keinen Sinn ergeben, sollte aber lesbar 
sein. An ihm messe ich die Lesbarkeit einer Schrift, ihre Anmutung,wie 
harmonisch die Figuren zueinander stehen und prüfe, wie breit oder schmal sie läuft. EinBlindtext sollte möglichst viele 
verschiedene Buchstaben enthalten und in der Originalsprachegesetzt sein. Er muss keinen Sinn ergeben, sollte aber lesbar 
sein.