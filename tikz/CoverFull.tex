%&../TikZPreamble
\ifdefined\INCLUDETIKZ
\else 
	% !TeX encoding = UTF-8
% !TeX spellcheck = de_DE
% !TeX root = ../thesis.tex
%-------------------------------------
%--- Abkuerzungen START
% Name, Symbol, Beschreibung
\newakronym{doi}{DOI}{Digital Object Identifier}
%--- Abkuerzungen ENDE
%-----------------------------------------------------
%--- Indizes START
% Name, Symbol, Beschreibung, Sortkey
\newindex{ist}{\text{ist}}{Istgröße}{ist}
\newindex{soll}{\text{soll}}{Sollgröße}{soll}
%--- Indizes ENDE
%--- Formelzeichen START
% Name, Symbol, Beschreibung, Siunit, Wert, Größe, Add, Sortkey
\newlatin{x}{\vec{x}}{Positionsvariable}{\meter}{}{3}{}{x}
\newlatin{err}{\vec{e}\vind{x}}{Positionsfehler}{\meter}{}{3}{}{ex}
\newgreek{phi}{\varphi}{Rotation um die z"=Achse}{\radian}{}{1}{}{u}
\newgreek{deltaphi}{\Delta\varphi}{Rotationsfehler um die z"=Achse}{\radian}{}{1}{}{du}
%--- Formelzeichen ENDE
%--- Erlaeuterungen:
%Sorkeys Greek
%a	alpha
%b	beta
%c	gamma
%d	delta
%e	epsilon
%f	zeta
%g	eta
%h	theta
%i	iota
%j	kappa
%k	lambda
%l	mu
%m	nu
%n	xi
%o	omicron
%p	pi
%q	rho
%r	sigma
%s	tau
%t	psilon
%u	phi
%v	chi
%w	psi
%x	omega
	\begin{document}
\fi
\begin{tikzpicture}[auto,font=\footnotesize]
% % % % % % % % % % % % % % % % % % % % % % % % %
\newcommand{\ruecken}{15.5563mm}
\newcommand{\aufschlag}{4.2464mm}

\node [inner sep=0, outer sep=0, anchor=center] (nullpunkt_back) at (0,0) {};
\node [inner sep=0, outer sep=0, anchor=center] (maxpunkt_back) at (210,297) {};
\node [inner sep=0, outer sep=0, anchor=center] (backtext) at (105,257) {};
\node [inner sep=0, outer sep=0, anchor=center] (qrcodepos) at (105,30) {};

\node [inner sep=0, outer sep=0, anchor=center] (nullpunkt_ruecken) at ($(210,0) + (0,0)$) {};
\node [inner sep=0, outer sep=0, anchor=center] (maxpunkt_ruecken) at ($(maxpunkt_back) + (\ruecken,0)$) {};

\node [inner sep=0, outer sep=0, anchor=center] (nullpunkt_front) at ($(210,0) + (\ruecken,0)$) {};
\node [inner sep=0, outer sep=0, anchor=center] (maxpunkt_front) at ($(maxpunkt_ruecken) + (210,0)$) {};

\node [inner sep=0, outer sep=0, anchor=center] (nullpunkt_aufschlag) at ($(nullpunkt_back) - (\aufschlag,\aufschlag)$) {};
\node [inner sep=0, outer sep=0, anchor=center] (maxpunkt_aufschlag) at ($(maxpunkt_front) + (\aufschlag,\aufschlag)$) {};

%\draw [draw=orange] (nullpunkt_back) rectangle (maxpunkt_back);
%\draw [draw=blue] (nullpunkt_ruecken) rectangle (maxpunkt_ruecken);
%\draw [draw=green] (nullpunkt_front) rectangle (maxpunkt_front);
%\draw [draw=pink] (nullpunkt_aufschlag) rectangle (maxpunkt_aufschlag);

\node [anchor=north east] (Coverbild) at ($ (maxpunkt_front) - (5,150) $) {};
%
\node [anchor=north east, minimum height=20mm, minimum width=60mm, fill=RWTHhellblau, text=weiss, text width=50mm, rotate=90, align=center] (Institut) at (maxpunkt_front -| nullpunkt_front) {\textsf{\institute}};
%
\node [anchor=north east, minimum height=15mm, text width=170mm, minimum width=190mm, align=right, fill=RWTHhellblau, text=weiss] at (Coverbild.south -| maxpunkt_front) (Name) {\LARGE \textsf{\autor}};
%
\node [anchor=north west, text width=150mm, minimum width=150mm, align=left, text=RWTHhellblau, font=\huge] at ($ (Name.south west) + (0, -15mm) $) (Titel) {\textsf{\titel}};

\node [anchor=north east, minimum height=\ruecken, minimum width=60mm, text=weiss, text width=50mm, rotate=90, align=center] (Institut_ruecken) at (maxpunkt_ruecken -| nullpunkt_ruecken) {\textsf{\institute}};

\node [anchor=center, text=RWTHhellblau, text width=150mm, rotate=90, align=center,font=\Large] (rueckentext) at ($(nullpunkt_ruecken)!0.5!(maxpunkt_ruecken)$) {\textsf{\titel}};

\node [anchor=west, text=RWTHhellblau, rotate=90, align=center,font=\Large] (rueckenautor) at ($(nullpunkt_ruecken) + (\ruecken/2,10)$) {\textsf{\autor}};

\tolerance		= 300 	% Weissraum / White space, default 1000
\hyphenpenalty = 250 %default 1000
\node [anchor=north, text width=145mm, align=justify, font=\normalsize] at (backtext) {\parskip=12pt \textsf{% !TeX encoding = UTF-8
% !TeX spellcheck = de_DE
% !TeX root = ../dissertation.tex

tldr, Windows: \texttt{setData.tex} ändern, \texttt{Script\_precompile.bat} ausführen, alle \texttt{*.tex}-Dateien in \texttt{tikz} kompilieren und anschließend \texttt{dissertation.tex} mit \texttt{latexmk} kompilieren.

tldr, Linux: \texttt{setData.tex} ändern, \texttt{Script\_create.sh} ausführen.

Latex entwickelt sich schnell weiter und viele Pakete, die in Vorlagen verwendet werden, sind inzwischen veraltet. Dementsprechend wollte ich für meine Dissertation eine aktuelle Vorlage erstellen. Zusätzlich wollte ich gerne ein automatisch sortiertes Symbolverzeichnis und nach und nach kamen weitere Anforderungen hinzu. Insgesamt entstand so eine Vorlage, die unter anderem über die folgenden Eigenschaften verfügt:

\begin{itemize}
	\item Vorkompilieren der Präambel zur Beschleunigung des Kompilierens.
	\item Layouts für die Einreichung (einseitig) und die finale Version (zweiseitg) in einer Datei.
	\item Automatisch sortiertes Symbolverzeichnis und Verlinkung im Text mit Angabe der Seitenzahl des ersten Vorkommens.
	\item Verlinkte Symbole im Text und in Bildern.
	\item Literaturverzeichnis mit Angabe der Seitenzahlen auf denen die Literaturstelle verwendet wird.
	\item Direkte Einbindung der TikZ-Dateien für die finale Version oder Einbindung der PDF-Dateien zur Beschleunigung des Kompilierens.
	\item Einhaltung von Empfehlung zur Typografie und zum Satz. Ausnahme bildet hier die Zeichen pro Zeile, die leicht zu hoch ist.
	\item Umgebunden für Abbildungen und Tabellen, die keine fließenden Umgebungen darstellen. Diese können im Anhang verwendet werden, wenn viele Abbildungen hintereinander eingefügt werden.
	\item Bash-Skripte zum Kompilieren unter Linux
\end{itemize}}};
%
\node at (qrcodepos) (qrcode) {\textcolor{schwarz}{\qrcode[hyperlink, height=2cm]{\DOIURL}}};
\node [below=0cm of qrcode, font=\normalsize, text=schwarz] {\textsc{Doi:} \nolinkurl{\DOI}};

\begin{scope}[on background layer]
	\draw [draw=none,fill=RWTHhellblau] (Institut.south west) rectangle (maxpunkt_aufschlag -| nullpunkt_ruecken);
	\draw [draw=none,fill=RWTHhellblau] (Name.south west) rectangle (Name.north east -| maxpunkt_aufschlag);
\end{scope}
% % % % % % % % % % % % % % % % % % % % % % % % %
\end{tikzpicture}
\ifdefined\INCLUDETIKZ
\else 
\end{document}
\fi