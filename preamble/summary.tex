% !TeX encoding = UTF-8
% !TeX spellcheck = en_US
% !TeX root = ../thesis.tex

tldr, windows: change \texttt{setData.tex}, run \texttt{Script\_precompile.bat}, compile all \texttt{*.tex}-files in \texttt{tikz} and finally run \texttt{thesis.tex} with \texttt{latexmk}.

tldr, linux: change \texttt{setData.tex}, run \texttt{Script\_create.sh}.

Latex is evolving rapidly and many packages used in templates are now obsolete. Accordingly I wanted to create a current template for my dissertation. Additionally I wanted an automatically sorted symbol directory and little by little more requirements were added. All in all, a template was created that has the following features, among others:

\begin{itemize}
	\item Pre-compile the preamble to speed up compilation.
	\item Layouts for the submission (one page) and the final version (two pages) in one file.
	\item Automatically sorted list of symbols and links in the text with the page number of the first occurrence.
	\item Linked symbols in text and images.
	\item Bibliography with indication of the page numbers on which the reference is used.
	\item Direct integration of TikZ-files for the final version or integration of PDF files to speed up compilation.
	\item Compliance with recommendations on typography and typesetting. The exception are the characters per line, which is slightly too high.
	\item Environments for figures and tables that do not represent floating environments. These can be used in the appendix if many figures are inserted one after the other.
	\item Bash-scripts for compilation using linux
\end{itemize}