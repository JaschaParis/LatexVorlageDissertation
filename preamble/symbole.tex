% !TeX encoding = UTF-8
% !TeX spellcheck = de_DE
% !TeX root = ../dissertation.tex
% Formula symbols
\chapter*{Formelzeichen und Abkürzungen}

\manualmark
\chead{}
\ohead{Formelzeichen und Abkürzungen}
\cfoot{}
\ofoot*{\pagemark}
\addcontentsline{toc}{chapter}{Formelzeichen und Indizes}

In dieser Arbeit wird die folgende Notation für die mathematische Darstellung verwendet: Vektoren erhalten einen Vektorpfeil, vergleiche Vektor \vec{v}, Matrizen werden groß und fett gedruckt, vergleiche Matrix \mat{A}, und komplexe Zahlen werden unterstrichen dargestellt, vergleiche komplexe Zahl \cx{z}. Sollen andere Darstellungen verwendet werden, kann die Definition von \texttt{\textbackslash vec} und \texttt{\textbackslash mat} in der Präambel verändert werden. 

\ifdefined\BUILDGLOSSARIES
	\setglossarysection{section}
	\printglossary[type=latin_symbols, style=symbols, title=Lateinische Formelzeichen]
	\newpage
	\printglossary[type=greek_symbols, style=symbols, title=Griechische Formelzeichen]
	\newpage
	\printglossary[type=indices, style=indices, title=Indizes]
	\newpage
	\printglossary[type=akronym, style=akronym, title=Abkürzungen]
\else 
\fi

\newpage\section*{Abbildungslegende}

Hier könnte noch eine Abbildungslegende eingefügt werden.