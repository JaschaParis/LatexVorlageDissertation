% !TeX encoding = UTF-8
% !TeX spellcheck = de_DE
% !TeX root = ../thesis.tex


\chapter{Mein Anhang}
\label{sec:App_HS_Optimierung}

Hier kann der Anhang eingefügt werden. Fließende Umgebungen in Latex verbessern den Satz und das Layout unter Wahl der richtigen Einstellungen extrem. Daher sollten Abbildungen auch immer mit solchen Umgebunden gesetzt werden. In einem Anhang, der u.U.\ sehr viele Abbildungen enthält, kann dies aber zu unerwünschten Effekten führen. Dazu führt diese Vorlage die \texttt{herefigure}-Umgebung ein, die eine Abbildung ohne Fließumgebung einbettet, siehe \cref{img:DissertationsprozessOhneFliessumgebung}. Das gleiche gibt es auch für Tabellen.

\begin{figure}[hbt]
	\centering
	\includeFig{Dissertationsprozess}{tikz/dissertationsprozess}
	\caption[Kurzunterschrift für Abbildungsverzeichnis]{Prozess zur Erstellung einer Dissertation im Anhang}
	\label{img:DissertationsprozessOhneFliessumgebung}
\end{figure}
